\section*{Zusammenfassung}

Krebszellen können sich vom Primärtumor lösen und das umgebende Gewebe abbauen. 
Kontinuierliche mathematische Modelle werden verwendet, um diesen Prozess besser zu verstehen.

In diesem Zusammenhang basieren die Modelle in der Regel auf mindestens drei Schlüsselkomponenten: den Tumorzellen, dem umgebenden Gewebe oder der extrazellulären Matrix (ECM) und den matrixabbauenden Enzymen (MDE). Diese Variablen werden in dem hier behandelten Modell in einem System partieller Differentialgleichungen gekoppelt, um den komplexen Prozess der Tumorinvasion zu beschreiben.

Aufgrund der hohen Anzahl der freien Parameter solcher Modelle, hat eine Konfiguration dieser Parameter hohen Einfluss auf die produzierten Ergebnisse. Diese Arbeit untersucht den Einfluss der Wahl dieser Konfigurationen, sowie den, der Dimension auf die Ergebnisse einer Simulation.

Um realitätsnahe biologische Szenarien zu simulieren, verlangen Simulationen mindestens zwei am besten drei räumliche Dimensionen.
Da in der Literatur fast ausschließlich eindimensionale Experimente beschrieben werden, untersuchen wir unser Modell nur in höheren Dimensionen: zwei- oder dreidimensional. Darüber hinaus wurde hauptsächlich die homogene Struktur der extrazellulären Matrix (ECM) behandelt, jedoch für eine heterogene extrazelluläre Matrix Struktur nur solche Fälle analysiert, die wenig biologische Relevanz haben. Die Struktur der epithelialen Schicht und der benachbarten extrazellulären Matrix ist jedoch in biologischem Gewebe deutlich organisierter als in den meisten später gezeigten Simulationen und anderen Beispielen aus der Literatur. Diese Organisiertheit kann zu erheblichen Veränderungen der Ergebnisse führen, selbst wenn die Parameter des Systems konstant gehalten werden.

Diese Arbeit soll einen Einstiegspunkt geben, um das zu grundliegende Modell für realistische Experimente zu nutzen. Die Parameteranalyse soll hierbei auch helfen eine angebrachte Wahl der Parameter zu erleichtern.

\clearpage
\section*{Abstract}

Cancer cells can detach from the primary tumor and degrade the surrounding tissue. 
Continuous mathematical models are been utilized for a better understanding of the process.

In this context the models typically rely on at least three key components: tumor cells, the surrounding tissue or extracellular matrix (ECM) and matrix-degrading enzymes (MDE). These variables are coupled in the investigated model in a  system of partial differential equations to describe the complex process of tumor invasion.

Due to such a model's high degree of freedom, the configuration of the model parameters significantly influences the resulting outcomes. This study examines the influence of choosing these configurations, as well as the dimensionality, on the results of a simulation.

To simulate realistic biological scenarios, simulations require at least two, preferably three, spatial dimensions. Since the literature predominantly describes one-dimensional experiments, we only investigate our model in higher dimensions: two or three dimensions. Furthermore, primarily, the homogeneous structure of the extracellular matrix (ECM) has been addressed, but for a heterogeneous extracellular matrix structure, only cases with little biological relevance have been analyzed. However, the structure of the epithelial layer and the adjacent extracellular matrix is significantly more organized in biological tissue than in most simulations shown later and other examples from the literature. This organization can lead to significant changes in the results, even when the parameters of the system are kept constant.

This work aims to provide a starting point for using the fundamental model for realistic experiments. The parameter analysis should also help facilitate an appropriate choice of parameters.
