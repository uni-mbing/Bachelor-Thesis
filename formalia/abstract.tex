\section*{Zusammenfassung}

Krebszellen können sich vom Primärtumor lösen und das umgebende Gewebe abbauen. 
Kontinuierliche mathematische Modelle wurden in der Vergangenheit mehrmals verwendet, um diesen Prozess besser zu verstehen.

In diesem Zusammenhang basieren die Modelle in der Regel auf mindestens drei Schlüsselkomponenten: den Tumorzellen, dem umgebenden Gewebe oder der extrazellulären Matrix (ECM) und den matrixabbauenden Enzymen (MDE). Diese Variablen werden in dem hier behandelten Modell in einem System partieller Differentialgleichungen gekoppelt um den komplexe Prozess der Tumor Invasion zu beschreiben.

Da ein solches Modell ein hohes Mas an Freiheitsgraden besitzt, hat eine Konfiguration der Parameter des Modells hohen Einfluss auf die produzierten Ergebnisse.

Diese Arbeit untersucht den Einfluss der Wahl dieser Konfigurationen sowie auch den der Dimension auf die Ergebnisse einer Simulation.

In der Literatur werden fast ausschlieslich eindimensionalen Experimente beschrieben, daher werden die Experimente hier nur in hoeheren Dimensionen durchgeführt, zwei oder drei dimensional. Darüber hinaus wurde hauptsächlich die homogene Struktur der extrazellulären Matrix (ECM) behandelt, jedoch fuer eine heterogene extrazelluläre Matrix Struktur nur solche Faelle analysiert, die wenig biologische Relevanz haben. Die Struktur der epithelialen Schicht und der benachbarten extrazellulären Matrix ist jedoch in biologischem Gewebe organisierter als in den meisten später gezeigten Simulationen und anderen Beispielen aus der Literatur. Diese Organisiertheit kann zu erheblichen Veränderungen der Ergebnisse führen, selbst wenn die Parameter konstant gehalten werden.

Um realitätsnahe biologische Szenarien zu simulieren, verlangen Simulationen mindestens zwei räumlichen Dimensionen. Daher soll diese Arbeit einen Einstiegspunkt geben um das zu grundliegende Modell fuer solche Simulationen zu nutzen. Die Parameteranalyse soll hierbei auch helfen eine angebrachte Wahl der Parameter zu erleichtern.

\clearpage
\section*{Abstract}

Cancer cells have the ability to detach from the primary tumor and degrade the surrounding tissue. Continuous mathematical models have been utilized in the past to better understand this process.

In this context, these models typically rely on at least three key components: tumor cells, the surrounding tissue or extracellular matrix (ECM), and matrix-degrading enzymes (MDE). These variables are coupled in a system of partial differential equations to describe the complex process of tumor invasion.

Due to the high degree of freedom of such a model, the configuration of the model parameters significantly influences the resulting outcomes. This study investigates the impact of choosing these configurations, as well as the dimensionality, on the results of a simulation.

In the literature, predominantly one-dimensional experiments are described, hence the experiments conducted here are performed in higher dimensions, two or three dimensional. Furthermore, while the homogeneous structure of the extracellular matrix (ECM) has been mainly addressed, cases involving a heterogeneous extracellular matrix structure have been analyzed to a lesser extent, despite their limited biological relevance.

However, the structure of the epithelial layer and the adjacent extracellular matrix in biological tissue is more organized than in most simulations and other examples presented in the literature. This organization can lead to significant changes in results, even when parameters are held constant.

To simulate realistic biological scenarios, simulations require at least two spatial dimensions. Therefore, this work aims to provide a starting point for utilizing the underlying model for such simulations. The parameter analysis aims to facilitate an appropriate selection of parameters.
