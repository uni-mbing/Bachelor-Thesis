\section*{Zusammenfassung}

Krebszellen können sich vom Primärtumor lösen und das umgebende Gewebe abbauen. 
Kontinuierliche mathematische Modelle wurden in der Vergangenheit mehrmals verwendet, um diesen Prozess besser zu verstehen. In diesem Zusammenhang basiert ide Modelle in der Regel auf mindestens drei Schlüsselkomponenten: den Tumorzellen, dem umgebenden Gewebe oder der extrazellulären Matrix (ECM) und den matrixabbauenden Enzymen (MDE). Das hier verwendete Modell beschreibt die obigen drei genannten Parameter, wobei Nullstrom Randbedingungen verwendet werden. \\ \\
Die Analyse dieses Modells wird in der Literatur größtenteils in $1D$ durchgeführt, und einzelne Beispiele wurden in $2D$ gemacht. Allerdings zeigen vorläufige Reproduktionen des Modells, dass höhere Dimensionen signifikant unterschiedliche 
Ergebnisse liefern.
Daher stellt sich die Frage, ob die Parameter für dieses Modell für Simulationen in $2D$ oder $3D$ unterschiedlich ausgewählt werden müssen oder ob die Ergebnisse und Analysen für den eindimensionalen Fall inkorrekt sind. \\ \\
Darüber hinaus wurde in der Literatur die heterogene Struktur der extrazellulären Matrix (ECM) bereits behandelt. Die Struktur der epithelialen Schicht und der benachbarten extrazellulären Matrix ist jedoch in biologischem Gewebe organisierter als in den gezeigten Simulationen und anderen Beispielen. Daher könnten einfachere Unterteilungen der Geometrie in ECM-Gewebe aussagekräftigere Ergebnisse liefern. \\ \\
Das Ziel dieser Arbeit ist es, einerseits die Parameter und das Modell für höhere 
Dimensionen zu untersuchen und andererseits eine einfache Heterogenität der ECM-Struktur 
in Betracht zu ziehen.

\clearpage
\section*{Abstract}
Cancer cells can migrate from the primary tumor and degrade the surrounding tissue. Continuous mathematical models have been used several times in the past to better understand this process. In this context, the model is usually based on at least three key species, the tumor cells, the surrounding tissue or extracellular matrix (ECM) and the matrix degradative enzymes (MDE). The investigated model in this work describes the above mentioned 3 parameters, with zero-flux boundry conditions. \\ \\
The analysis of this models is mostly done in $1D$ in the literature and individual examples were done in 2D. However, reproductions of the model show that higher dimensions produce significantly different results. The question therefore arises as to whether the parameters for this model need to 
be selected differently for simulations in 2D or 3D, or whether the results and analysis for the one dimensional case is incorrect. \\
Ergebnis einfuegen\\ \\
Furthermore, the heterogeneous ECM structure has been addressed in the literature. However, the structure of the epithelial layer and the adjacent extracellular matrix is more organized in biological tissue than in the simulations shown in and, for example. herefore, simpler subdivisions of the geometry into ECM tissue could provide more meaningful results. \\ \\
The aim of this work is to investigate the parameters and the model for higher dimensions on the one hand, and to consider a simple heterogeneity of the ECM structure on the other hand.