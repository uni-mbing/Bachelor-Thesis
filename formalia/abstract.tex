\section*{Zusammenfassung}

Krebszellen können sich vom Primärtumor lösen und das umgebende Gewebe abbauen. 
Kontinuierliche mathematische Modelle wurden in der Vergangenheit mehrmals verwendet, um diesen Prozess besser zu verstehen.\\
In diesem Zusammenhang basieren die Modelle in der Regel auf mindestens drei Schlüsselkomponenten: den Tumorzellen, dem umgebenden Gewebe oder der extrazellulären Matrix (ECM) und den matrixabbauenden Enzymen (MDE). Diese Variablen werden in dem hier behandelten Modell in einem System partieller Differentialgleichungen gekoppelt um den komplexe Prozess der Tumor Invasion zu beschreiben.\\
Da ein solches Modell ein hohes Mas an Freiheitsgraden besitzt, hat eine Konfiguration der Parameter des Modells hohen Einfluss auf die produzierten Ergebnisse.\\
Diese Arbeit untersucht den Einfluss der Wahl dieser Konfigurationen sowie auch den der Dimension auf die Ergebnisse einer Simulation.\\ 
In der Literatur werden fast ausschlieslich eindimensionalen Experimente beschrieben, daher werden die Experimente hier nur in hoeheren Dimensionen durchgeführt, zwei oder drei dimensional. Darüber hinaus wurde hauptsächlich die homogene Struktur der extrazellulären Matrix (ECM) behandelt, jedoch fuer eine heterogene extrazelluläre Matrix Struktur nur solche Faelle analysiert, die wenig biologische Relevanz haben. Die Struktur der epithelialen Schicht und der benachbarten extrazellulären Matrix ist jedoch in biologischem Gewebe organisierter als in den meisten später gezeigten Simulationen und anderen Beispielen aus der Literatur. Diese Organisiertheit kann zu erheblichen Veränderungen der Ergebnisse führen, selbst wenn die Parameter konstant gehalten werden.\\
Um realitätsnahe biologische Szenarien zu simulieren, verlangen Simulationen mindestens zwei räumlichen Dimensionen. Daher soll diese Arbeit einen Einstiegspunkt geben um das zu grundliegende Modell fuer solche Simulationen zu nutzen. Die Parameteranalyse soll hierbei auch helfen eine angebrachte Wahl der Parameter zu erleichtern.

\begin{comment}
Krebszellen können sich vom Primärtumor lösen und das umgebende Gewebe abbauen.
Kontinuierliche mathematische Modelle wurden in der Vergangenheit mehrmals verwendet, um diesen Prozess besser zu verstehen. In diesem Zusammenhang basieren die Modelle in der Regel auf mindestens drei Schlüsselkomponenten: den Tumorzellen, dem umgebenden Gewebe oder der extrazellulären Matrix (ECM) und den matrixabbauenden Enzymen (MDE). \\ \\
Diese Arbeit untersucht den Einfluss der freien Parameter des Modells als auch den der Dimension auf die Ergebnisse der Simulationen. Da in der Literatur fast ausschlieslich eindimensionalen Versuche durchgeführt wurden, werden die Experimente hier nur in hoeheren Dimensionen durchgeführt. \\ \\
Darüber hinaus wurde in der Literatur die homogene Struktur der extrazellulären Matrix (ECM) bereits behandelt, jedoch fuer eine heterogene extrazelluläre Matrix Struktur nur solche Faelle analysiert, die wenig biologische Relevanz haben. Die Struktur der epithelialen Schicht und der benachbarten extrazellulären Matrix ist jedoch in biologischem Gewebe organisierter als in den später gezeigten Simulationen und anderen Beispielen. \\ \\
Das Ziel dieser Arbeit ist es, einerseits die Parameter und das Modell für höhere Dimensionen zu untersuchen und andererseits eine einfache Heterogenität der ECM-Struktur in Betracht zu ziehen. Einige Parameter haben gewichtigeren Einfluss auf die Ergebnisse als andere, zudem variieren die Groessenordnungen und damit auch die Einflussbereiche der Variablen stark. Eine beretis einfache heterogene Struktur der extrazellulären Matrix anzunehmen veraendert die Ergebnisse ebenfalls stark, da diese entscheidend ist fuer die Bewegung der Tumorzellen im Gewebe.
\end{comment}


\clearpage
\section*{Abstract}

Cancer cells have the ability to detach from the primary tumor and degrade the surrounding tissue. Continuous mathematical models have been utilized in the past to better understand this process.\\
In this context, these models typically rely on at least three key components: tumor cells, the surrounding tissue or extracellular matrix (ECM), and matrix-degrading enzymes (MDE). These variables are coupled in a system of partial differential equations to describe the complex process of tumor invasion.\\
Due to the high degree of freedom of such a model, the configuration of the model parameters significantly influences the resulting outcomes. This study investigates the impact of choosing these configurations, as well as the dimensionality, on the results of a simulation.\\
In the literature, predominantly one-dimensional experiments are described, hence the experiments conducted here are performed in higher dimensions, two or three dimensional. Furthermore, while the homogeneous structure of the extracellular matrix (ECM) has been mainly addressed, cases involving a heterogeneous extracellular matrix structure have been analyzed to a lesser extent, despite their limited biological relevance.\\
However, the structure of the epithelial layer and the adjacent extracellular matrix in biological tissue is more organized than in most simulations and other examples presented in the literature. This organization can lead to significant changes in results, even when parameters are held constant.\\
To simulate realistic biological scenarios, simulations require at least two spatial dimensions. Therefore, this work aims to provide a starting point for utilizing the underlying model for such simulations. The parameter analysis aims to facilitate an appropriate selection of parameters.

\begin{comment}
Cancer cells can migrate from the primary tumor and degrade the surrounding tissue. Continuous mathematical models have been used several times in the past to better understand this process. In this context, the model is usually based on at least three key species, the tumor cells, the surrounding tissue or extracellular matrix (ECM) and the matrix degradative enzymes (MDE). The investigated model in this work describes the above mentioned 3 parameters, with zero-flux boundry conditions. \\ \\
The analysis of this models is mostly done in $1D$ in the literature and individual examples were done in 2D. However, reproductions of the model show that higher dimensions produce significantly different results. The question therefore arises as to whether the parameters for this model need to 
be selected differently for simulations in 2D or 3D, or whether the results and analysis for the one dimensional case is incorrect. \\
Ergebnis einfuegen\\ \\
Furthermore, the heterogeneous ECM structure has been addressed in the literature. However, the structure of the epithelial layer and the adjacent extracellular matrix is more organized in biological tissue than in the later shown simulations shown and other examples. Therefore, simpler subdivisions of the geometry into ECM tissue could provide more meaningful results. \\ \\
The aim of this work is to investigate the parameters and the model for higher dimensions on the one hand, and to consider a simple heterogeneity of the ECM structure on the other hand. As you will see some parameters have a stronger impact on the results than others, with also varying scale and reach of influence.
\end{comment}