\section{Method}

\subsection{Software and Implementation}
\subsection{Systematic Parameter Analysis}
\subsubsection{2D}
\subsubsection{3D}
\subsubsection{Non-heterogenous ECM Structure}

This work will investigate how the results of the modified model in the modelling section will be affected be varying free parameters $\lambda, \delta, \mu_1, \mu_2, \mu_3$ as well as how the chosen dimension for a simulation will influence the output.
For this we will start with comparing results first within the same dimension, varying the free parameters. Conducting these experiments we hope that we can see a pattern occuring, which is shared across one dimension. After this we will investigate how the results are changed when we change the dimension but keep the free parameters the same. 


\begin{center}
\begin{tabular}{|| c | c | c | c | c || c | c | c | c | c || c | c | c | c | c ||}
    \hhline{||=|=|=|=|=||=|=|=|=|=||=|=|=|=|=||}
    1D \small & & & & & 2D & & & & & 3D & & & & \\
    \hhline{||=|=|=|=|=||=|=|=|=|=||=|=|=|=|=||}
    $\lambda$ & $\delta$ & $\mu_1$ & $\mu_2$ & $\mu_3$  & $\lambda$ & $\delta$ & $\mu_1$ & $\mu_2$ & $\mu_3$ & $\lambda$ & $\delta$ & $\mu_1$ & $\mu_2$ & $\mu_3$ \\ 
    \hline
    a & b & c & d & e & a & b & c & d & e & a & b & c & d & e \\ \hline 
    a & b & c & d & e & a & b & c & d & e & a & b & c & d & e \\ \hline 
    a & b & c & d & e & a & b & c & d & e & a & b & c & d & e \\ \hline 
    a & b & c & d & e & a & b & c & d & e & a & b & c & d & e \\ \hhline{||=|=|=|=|=||=|=|=|=|=||=|=|=|=|=||} 

\end{tabular}
\end{center}

When comparing the effects of the free parameters on the simulation we need to keep in mind, that some of them are by assupmtion fixed, whilst others are defined within a reasonable range and again others have no restrictions at all, for those we assume the value ranges in the modelling section. At this first stage of the investigation we change only one parameter within each group of experiments, for example we only study how the output changes when $\delta$ changes, we do this with every parameter and compare the results within their dimension. When we are done with that we compare the results with changing dimension whilst keeping every other free parameter equal. \newline
With the core question that if different dimensions yield different results, we hope to find characteristics prevailing in each dimension, but mainly concentrate on the features that arise with the same choice of parameters varying the dimension. We hope to find answers to which changes in outcome emerge and how to explain the different results. \newline
As a last step this work compares the found results with clinical results, from in-vitro results. Why in-vitro, because this models lacks the biological complexity which governs many effects regarding tumor growth in living bodies, which would yield to quite different behaviour.