\section{Introduction}
Modelling tumor growth plays a key role in understanding the complex mechanisms, governing development and progression of cancer diseases. Since cancer is one of the leading death causes worldwide and many of its forms are incurable, challenges in the area of Oncology require researchers to have a deep understanding in as well the biological foundation, which lead to malignant cell mutation and factors for tumor growing and spreading, as well as the mathematical models used for simulating these events. This Bachelorthesis is dedicated to analyse Anderson et al.'s~\cite{anderson_continuous_1998,anderson_mathematical_2000} model for tumor modelling.\newline
The dynamics of tumerous growth are an intricate system, which is influenced by numerous biological and chemical factors, as well as genetic pre-dispostitions, the surrouding tissue of cancer cells, angiogene processes an interactions with the immune system. The integration of these factors in mathematical models allow us to decode these complex interactions with quantification and help us understand the fundamental mechanisms, which surroud cancerous diseases. \newline 
Mathematical models are a very important part in Oncology. They are used to quantify biological phenomena and therefore help to predict and understand tumor development and treatment response. In Mathematical Oncology we differentiate between continuous, discrete and hybrid models~\cite{BEKISZ2020101198}. For the continuous type, cells and tissue are described over time with partial differential equations modelling continuous quantities like in our case the cell or extracellular matrix density. In the discreate case, a entity based model is used, pursued with the goal to better understand the phenoma on cell level. This approach allows the researcher to better implement biological effects a cell has with its outer circumstances, like interaction with other cells, nutrients or other microorganisms. As the name implies use these models discrete values to describe the temporal course of events. Hybrid models try to combine both approaches, to offer efficient systems capturing cell level events as well as continuous changes in outer circumstances.\newline 
In this work we are investigating how a continuous model proposed by Anderson et al.~\cite{anderson_continuous_1998,anderson_mathematical_2000} to analyse tumur development in the early stages performs in the case of different dimensions and free parameter values. The model examines the first two stages of a cancer disease; tumor initiaition, where the tumor cells are localized to a small area and have not yet spread throughout the body; and tumor promotion, with the tumor cells growing and proliferating, invading the surrounding tissue~\cite{10.1158/2159-8290.CD-21-1059}. From examples of the original paper we can already see that the model's results vary with the dimensionality of the space we are modelling the partial differential equations in. Our main focus lies on comparing simulations of two dimensions with those of three dimensions of extracellular matrix invasion by the tumor growth. Additionally to the variation of dimensions we will have a closer look on how the geometry of the extracellular matrix will influence the tumor development. \newline 
Another point of interest is the investigation of how the model's free parameters influence the tumor dynamics growth. An important task is to give those parameters a biological meaning and to eventually gain insight into how to adjust them to make the simulation more realistic.