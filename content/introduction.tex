\section{Introduction}
Modeling tumor growth plays a cruical role in understanding the complex mechanisms governing the development and progression of cancer diseases. Since cancer is one of the leading death causes worldwide and many of its forms are incurable, challenges in the area of Oncology require researchers to have a deep understanding in the biological foundation, which leads to malignant cell mutation and factors for tumor growth and spreading, as well as the mathematical models used for simulating these events. This Bachelor thesis is dedicated to analyzing Anderson et al.'s~\cite{anderson_continuous_1998,anderson_mathematical_2000} model for modeling tumor invasion.

The dynamics of tumorous growth are an intricate system influenced by numerous biological and chemical factors, genetic pre-dispositions, the surrounding tissue of cancer cells, angiogenic processes and interactions with the immune system. Integrating these factors in mathematical models allows us to decode these complex interactions with quantification and helps us understand the fundamental mechanisms surrounding cancerous diseases.

Mathematical models are an essential part of Oncology; they are used to quantify biological phenomena and help to predict and understand tumor development and treatment response. In Mathematical Oncology, we differentiate between continuous, discrete and hybrid models~\cite{BEKISZ2020101198}. 

For the continuous type, cells and tissue are described over time with partial differential equations modeling continuous quantities like, in our case, tumor cell density, extracellular matrix concentration or matrix-degrading enzyme concentration.

In the discrete case, an entity-based model is used, pursued with the goal of better understanding the phenomena on the cell level. This approach allows the researcher to better implement a cell's biological effects with its outer circumstances, like interaction with other cells, nutrients or other microorganisms. As the name implies, use these models' discrete values to describe the temporal course of events.

Hybrid models try to combine both approaches to offer efficient systems capturing cell level events and continuous changes in outer circumstances.

This work investigates how a continuous model proposed by Anderson et al.~\cite{anderson_continuous_1998,anderson_mathematical_2000} to analyze tumor development in the early stages of a cancer disease performs in the case of different dimensions and free parameter values. The model examines the third and fourth stages of a cancer disease: tumor progression, where the tumor cells grow more extensive and take on more aggressive behavior due to further genetic instabilities and tumor invasion, with the tumor cells gaining the ability to penetrate and invade the surrounding tissue~\cite{10.1158/2159-8290.CD-21-1059}. 

From examples of the original paper, we can already see that the model's results vary with the dimensionality of the space, we are modeling the partial differential equations in. 

Our primary focus is investigating how the model's free parameters influence tumor dynamics growth. An important task is to give those parameters a biological meaning and eventually gain insight in how to adjust them to make the simulation more realistic. Another point of interest is comparing simulations of two dimensions with those of three dimensions of extracellular matrix invasion by tumor growth. Additionally to the variation of dimensions, we will briefly examine how the geometry of the extracellular matrix will influence tumor development.