\externaldocument{modelling}

\section{Experiments and Results}
For all the plots of the experiments the red curve indicates the tumour cell density, the blue curve the ECM density and the green curve the MDE concentration. In all of the experiments we used the value of $\epsilon = 0.01$ to match the inital conditions from \cite{anderson_mathematical_2000} and \cite{Kolev2010}.
\subsection{Two dimensional Results}
\subsubsection{Replicating results}
We will start with replicating previous results from Anderson et al.\cite{anderson_mathematical_2000}, Figure 1, making our curves fit the findings in their diagramms. 

%naming convention of figures: d0_d1_d2_eta_alpha_beta_gamma_mu1_mu2

\begin{figure}[t]
    \centering
    \includegraphics[width=\textwidth]{resources/images/2D_1e-3_1e-3_1e-3_10_0.1_0_0.005_1e-2_10_plot.png}
    \caption{Caption}
    \label{fig:2D_1e-3_1e-3_1e-3_10_0.1_0_0.005}
\end{figure}
In the first simulation the same parameter were used as in Anderson et al. first one dimensional experiment; $d_c = 0.001, d_m = 0.001, \gamma = 0.005, \eta = 10, \alpha = 0.1, \beta = 0, \mu_1 = 0, \mu_2 = 0$. Figure~\ref{fig:2D_1e-3_1e-3_1e-3_10_0.1_0_0.005} shows 4 snapshots of different points in time of tumour cell density, extracellular matrix density and matrix degrading enzymes concentration. In the conducted experiments it was shown that for every step in time done in the original paper we have done 4 steps, this is the reason for our time scale. Starting from the inital values seen at $t=0$ we see that after four time steps a very small unevenness has formed for the tumour cell density at $x\approx 0.1$. Both concentrations of MDEs and ECM have decreased as expected, looking at the model, MDEs have also invaded into the surrouding tissue, stretching the initial concentration around the origin.
In the next image showing the simulation after 40 timesteps we see that this unevenness has been propagated to form a hill at the leading edge of the tumour cells invading the surrounding tissue, at $x\approx 0.28$. MDEs also continued their diffusion into the area, decaying the ECM in their wake, decreasing them further. 
The last image, after 80 simulation time steps, we see that as well the hill that has formed at the leading edge of the tumour cells as well as the concentration of tumour cells at the origin, has decreased, due to the diffusion factor and the haptotactic flux. If we were to look at the simulation at later points in time, this curve will flatten even more, since with more time the ECM will be decayed and therefore the haptotactic flux coefficient $\gamma$ will lose its influence, leaving the movement of the cells to diffusion only. The curve for the MDEs has also flattened, yet not as strongly as the tumour cells concentration and as the observed before the ECM decayed where the MDEs were previously.\newline
Comparing \ref{fig:2D_1e-3_1e-3_1e-3_10_0.1_0_0.005} to figure 1 in \cite{anderson_mathematical_2000}, we can see major differences. The first image showing $t=0$ looks the same, which confirms that both experiments start with the same initial values. In the images showing the simulation at the second time checkpoint we see that though the tumour concentration and ECM density values are approximately the same, the MDE concentration is slightly lower in our experiment, which will get more pregnant in the later images. The unevenness having formed at the leading edge of the tumour cell concentration also looks to be slightly smaller. The differences in the third image are more strikingly, both $c$ and $m$ have considerably lower concentrations, yet the ECM value looks to in line. In our case the diffusion of the tumour cells into the tissue also seems to happen a littel bit too fast. The last time checkpoints manifest our findings, showing the same behaviour with ECM being approximately the same, tumour cell density and MDE concentration being clearly lower in our experiment and invasion of tissue happening too fast, leaving the lump at the origin $x=0$ too small. \newline 
This first of all confirms the initial supposition that with changing the dimension for the simulations the results also vary. We will now adjust the parameters iteratively to align our results with above compared experiment. For this we will now start with varying the MDE production coefficient $\alpha$, to get higher concentration values, and also change the diffusion terms $d_c$ and $d_m$, to adjust the pace of the invasion of tumour cells and MDEs into the area.

\begin{figure}[t]
    \centering
    \includegraphics[width=\textwidth]{resources/images/alpha_comparison.png}
    \caption{Caption}
    \label{fig:alpha_comparison}
\end{figure}
Figure~\ref{fig:alpha_comparison} shows a comparison of different values set for $\alpha$ and their effect on the curve of the MDE concentration. To determine the correct value for $\alpha$ to replicate the result we can look at the later points in time, $t=40$ and $t=80$ and see that with the value of $\alpha$ between $0.3$ and $0.4$ we will get a good approximation. The value of the original paper is at this point in time about $0.7$ at $x=0$. Fine tuning this parameter led us to $\alpha=0.35645$.

\begin{figure}[t]
    \centering
    \includegraphics[width=\textwidth]{resources/images/dc_comparison.png}
    \caption{Caption}
    \label{fig:dc_comparison}
\end{figure}

\begin{figure}[t]
    \centering
    \includegraphics[width=\textwidth]{resources/images/gamma_comparison.png}
    \caption{Caption}
    \label{fig:gamma_comparison}
\end{figure}

The next main difference was that the tumour cell invasion of the tissue looked slightly different, for this we also conducted a set experiments regarding the parameters $d_c$ and $\gamma$, seen in~\ref{fig:dc_comparison} and~\ref{fig:gamma_comparison}, showing in each comparison only the curve for the tumour cell concentration. Here we see that changing $d_c$ with equally distanced parameters does not give us an as regular result as it did regarding $\alpha$. Looking at the two later points in time, we see that the dashed curve with $d_c=0.0005$ resembles the original experiment really well. For the comparison of the parameter $\gamma$ we see that with increasing it form $0.005$ to $0.007$ the initial value was already really close to the to be replicated experiment, yet a slight adjustment to $\gamma = 0.0055$ worked better since we want the small hill at the leading edge of the tumour cell concentration in the latter two points in time to ber a little farther out, yet not as steep as for example $\gamma = 0.007$. \newline 
\begin{figure}[t]
    \centering
    \includegraphics[width=\textwidth]{resources/images/2D_5e-4_1e-3_1e-3_10_0.35645_0_0.0055_1e-2_10_plot.png}
    \caption{Caption}
    \label{fig:2D_5e-4_1e-3_1e-3_10_0.35645_0_0.0055}
\end{figure}
These adjustments leave with the final configuration for replicating the system with the curves in figure~\ref{fig:2D_5e-4_1e-3_1e-3_10_0.35645_0_0.0055} and the parameter settings also seen in the same figure.
Slightly adjusting the haptotatic flux to $\gamma=0.0055$ yields the following results, seen in figure~\ref{fig:2D_5e-4_1e-3_1e-3_10_0.35645_0_0.0055}. comparing our final version with the original one we can see that in the second point in time, at $t=4$ in our case, the values of the three curves at the $x=0$ are nearly the same. In the original experiment the bump in the curve for the tumour concentration looks more pregnant, but this is only due to the fact, that this experiment was most likely done on the unit line, not the unit cube, and therefore the x-scale has been streched to $y_{max}=1$ where in our case it is $x_{max} = 0.5$. The two later points in time confirm the similarity with having also nealy the same values for the three curves at $x=0$ but also their respective propagations in time look to be in line with the original experiment. 


\begin{comment}
a small cluster has formed at the leading edge of the tumour cells, this is due to the haptotactic flux having a higher value of 0.005 instead of 0.001 for random motility. This pulls the tumour cells towards the direction where the ECM is highest, into direction of the gradient of $e$. With passing time this cluster migrates furhter from the main body of the tumour outwards into the surrounding tissue. Comparing this with the diagramm from Anderson et al. there are major differences, for example in our experiment there is no cluste of cells remaining at the origin, in the diagramm on at point x = 0, which implies that our tumor has more or less completely dissolved into the tissue, which stands in contrast to the biological mechanism. So for the next experiment we will change the diffusion coefficients $d_c$ and $d_m$.  


% wrong parameters change later
\begin{figure}
    \centering
    \includegraphics[width=\textwidth]{resources/images/0.001_0.001_0.001_10_0.1_0_0.01_0_0.png}
    \caption{Caption}
    \label{fig:0.001_0.001_0.001_10_0.1_0_0.01_0_0}
\end{figure}

The second experiments, figure~\ref{fig:0.001_0.001_0.001_10_0.1_0_0.01_0_0}, has an increased value of $\gamma = 0.01$ and is for the remaing values similiar to experiment 1. Because of the larger haptotatic flux $\gamma$ the cluster at the leading edge of the tumour cells density is noticably higher than in the first experiment, the cells migrating faster into the tissue. This results in an also faster decay of the ECM density. In the last image at $t = 30$ we can clearly see that for $gamma = 0.01$ the MDE curve is considerably higher than both ECM density and the MDE concentration curve of experiment 1. 

\begin{figure}
    \centering
    \includegraphics[width=\textwidth]{resources/images/0.001_0.001_0.001_10_0.1_0_0.001_0_0.png}
    \caption{Caption}
    \label{fig:0.001_0.001_0.001_10_0.1_0_0.001_0_0}
\end{figure}

Figure~\ref{fig:0.001_0.001_0.001_10_0.1_0_0.001_0_0} shows the results of a simulation where $\gamma = 0.001$ is decreased and has similiar values for the other parameters as experiments 1,2. The cluster invading the tissue is almost not visible anymore and the migration process of the tumour cells happens, as was expected, slower than in the previous experiments. In the last image of this experiment at $t = 30$ we can see that if we compare it to the previous two experiments that ECM density is still higher than the MDE concentration, with the MDEs not having degraded the ECM structure this far outward from the origin. \newline
Drawing a first conclusion for the parameter of $\gamma$ we can observe that depending on it, as the name haptotactic flux coefficient also implies, the pace of the invasion of the surrounding tissue cruically depends on this and with this also the degradation of the moving farther away from the origin. This parameter also determines the scale of how well the tumour cells can invade the surrounding tissue, if this value low we see that there is almost no cluster leading at the edge, if the values is higher the cluster is more distinct which increases the ability to start the metastatic cascade.\newline
\begin{figure}
    \centering
    \includegraphics[width=\textwidth]{resources/images/0.001_0.001_0.001_10_0.1_0.5_0.005_0_0.png}
    \caption{Caption}
    \label{fig:0.001_0.001_0.001_10_0.1_0.5_0.005_0_0}
\end{figure}
In the next experiment, depicted in figure~\ref{fig:0.001_0.001_0.001_10_0.1_0.5_0.005_0_0} the parameter $\beta = 0.5$ is changed leaving the rest the same. This introduces a decay term in the MDE concentration. Here we can also see the cluster forming at the edge, and also propagating towards the gradient of the ECM, but in contrast to the first experiment we see that the ECM is not temporaily degraded, being stable over time. Also the MDE concentration stays constant. These two observations of MDE concentration and ECM density suggest that the MDE decay term and the production term for MDEs coming from the tumourous cells stays in balance, over time both curves dont seem change dramatically, having the same final configuration as the initail condition. The movement of the tumour cells looks generally the same with the difference to the first experiment that the pace at which the cells migrate outwards into the surrounding area is slower, this only makes sense, since the ECM is constant and therefore the gradient term of it nullifies itself. With time the tumour cells migrate to the outer limit of the unit square and take on a constant distribution across space. 
\begin{figure}
    \centering
    \includegraphics[width=\textwidth]{resources/images/0.001_0.001_0.001_20_0.1_0_0.002_0_0.png}
    \caption{Caption}
    \label{fig:0.001_0.001_0.001_20_0.1_0_0.002_0_0}
\end{figure}
In this last replicated experiment, shown by figure~\ref{fig:0.001_0.001_0.001_20_0.1_0_0.002_0_0} the ECM degrading rate is increase with $\eta = 20$ and the haptotactic flux coefficient is decreased to a value of $\gamma = 0.002$. The adjustment of the haptotactic coefficient is noticable, the cluster forming at the leading edge of the tumour cells being as striking like in the first experiment. Where in the first experiment the value of the tumour density between the cluster and the origin, was visibly lower than the cluster itself, we can see no such behaviour here. The bump is distinguishable but in direction of the origin the curve does not flatten, on the contrary it is about 1.5 times hihger than the cluster itself. It is interesting to see, that although the degradation rate of the ECM is higher, form a temporaily point of view, the actual degradation does not happen faster than in experiment 1, this is due to the fact, the the haptotactic flux coefficient is decreased, the tumour cells don't migrate as fast into the surrouding tissue and therefore do not get as fast into contact with the ECM as in experiment 1. This interplay makes the ECM curve look over time like the one from experiment 1, althought the parameters for the experiments are noticably different. The MDE concentration looks like in experiment 1, though at the first pionts in time of the snapshots the concentration is higher towards the origin, since in this area are also more tumour cells producing it. \newline 
In all of the experiments except where decay and production rate of the MDE seemed to be in balance you can see that over time the MDE fulfill their task, degrading the ECM, which in some in point in time sooner or later results in higher values for the MDE than the ECM everywhere. \newline

%insert replications of model with proliferation
\end{comment}

\subsubsection{Parameter Analysis}
From the replicated results shown in figures~\ref{fig:0.001_0.001_0.001_10_0.1_0_0.005_0_0}-$\textcolor{red}{insert correct last replicated result}$ we saw variation across all parameters except $d_c$ since this one is assumed to be constant, as described in Anderson et al.~\cite{anderson_mathematical_2000}. We are now going to take the baseline values of $(d_c, \gamma, \mu_1, \eta, \mu_2, d_m, \alpha, \beta) = (0.001, 0.005, 0.3, 10, 0.3, 0.001,0.1, 0.1)$ and start experimenting with it as seen in table~\ref{tab:systematic_analysis}, for this.
\subsection{Three Dimensional Results}
\subsubsection{Replicating Results}
\subsubsection{Parameter Analysis}
\subsection{Three Dimensional Simulations with Heterogenous ECM Structure}