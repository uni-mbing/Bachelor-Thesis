\externaldocument{modelling}

\section{Experiments and Results}
For all the plots of the experiments the red curve indicates the tumour cell density, the blue curve the ECM density and the green curve the MDE concentration. In all of the experiments we used the value of $\epsilon = 0.01$ to match the inital conditions from \cite{anderson_mathematical_2000} and \cite{Kolev2010}. \newline 
Mathematical Intuition of the three curves and how the parameters interact.


\subsection{Two dimensional Results without Proliferation}
\subsubsection{Replicating results}
We will start with replicating the experiment from  Anderson et al.\cite{anderson_mathematical_2000}, Figure~\ref{fig:2D_1e-3_1e-3_1e-3_10_0.1_0_0.005}, trzing to make our curves fit the findings in their diagramms. 

%naming convention of figures: d0_d1_d2_eta_alpha_beta_gamma_mu1_mu2

\begin{figure}[h]
    \centering
    %\includegraphics[width=\textwidth]{resources/images/2D_1e-3_1e-3_1e-3_10_0.1_0_0.005_1e-2_10_plot.png}
    \includegraphics[width=0.9\textwidth]{resources/images/first_replication_results.png}
    \caption{Images on left side 2D plots of the experiment, images on right side results produced by applying plot over line tool}
    \label{fig:2D_1e-3_1e-3_1e-3_10_0.1_0_0.005}
\end{figure}

We therefore used the same parameters as in Anderson et al.s first one dimensional experiment with the values; $d_c = 0.001, d_m = 0.001, \gamma = 0.005, \eta = 10, \alpha = 0.1, \beta = 0, \mu_1 = 0, \mu_2 = 0$. Figure~\ref{fig:2D_1e-3_1e-3_1e-3_10_0.1_0_0.005} shows these results for four different points in time. Since our experiments are preformed in two dimensions you can see on the left side the two dimensional plots of the tumour cell density and on the right side you can see plots produced by applying the Plot Over Line tool confiugrated like described in the method section, where there is not only the tumour cell density visible but also the matrix decaying enzyme concentration and the extra cellular matrix concentration. We choose these points in time, because we used a different timescale than Anderson et al. and as you will later see, our time points capture the effects that can be observed studying their results quite well, which implies that for every time step Anderson et al. did we had to do 4. The problem with the two diemnsional plots is that as you can see overlazing the different variables of tumour cell density, MDE concentration and ECM concentration, makes the plots unreadable, so you can only show one at a time, additionally to this it is not possible to estimate values for c, e, m at different locations in space and time. These problems are solved using the plot over line tool, as you can see in the plots on the right side the curves for c, e, m are clearly distinguishable and we can estimate their values at locations. This is why for most experiments we resort to using the plot over line plots instead of full 2D simulation images, they make evaluating and comparing the respective experiments way easier and still capture for the most part all occuring effects. 
Starting from the inital values at $t=0$ we see that after four time steps a very small unevenness has formed for the tumour cell density at $x\approx 0.1$. Diffusion and Haptotaxis have stretched the curve for the tumour cells as did diffusion for the MDEs. The ECM has clearly been decayed at the origin.
The next image shows the simulation after 40 timesteps; we see that the unevenness of the tumour cell concentration of the previous point in time has been propagated to form a small hill at the leading edge of the tumour cells invading the surrounding tissue, at $x\approx 0.28$, this effect is due to the haptotatic influence, which pulls the tumour cells further into the accessible area towards the gradient of c grad e, creating a seperation for the tumour cell density, where the other part is still oriented towards the origin. MDEs also continued their diffusion into the area, decaying the ECM in their wake, decreasing them further. 
In the last image, after 80 simulation time steps, we see that as well the hill that has formed at the leading edge of the tumour cells as well as the concentration of tumour cells at the origin, have continued flattenting and taking on a constant concentration throughout space, though we can still clearly distinguish both areas. If we were to look at the simulation at later points in time, this curve will flatten even more, since with more time the ECM will be decayed and therefore the haptotactic flux coefficient $\gamma$ will lose its influence, leaving the movement of the cells to diffusion only. The curve for the MDEs has also flattened, yet not as strongly as the tumour cells concentration and as the observed before the ECM decayed where the MDEs were previously. Thez will continue to do so, decaying the ECM and their concentration over time will increase due to no limiting factors in this experiment and on-going production contributed by the tumour cells c.\newline
Comparing \ref{fig:2D_1e-3_1e-3_1e-3_10_0.1_0_0.005} to figure 1 in \cite{anderson_mathematical_2000}, we can see major differences. The first image showing $t=0$ looks the same, which confirms that both experiments start with the same initial values condition. In the images showing the simulation at the second time checkpoint we see that though the tumour concentration and ECM density values are approximately the same, the MDE concentration is slightly lower in our experiment, which will get more pregnant in the later images. The unevenness having formed at the leading edge of the tumour cell concentration also looks to be slightly smaller. The differences in the third image are more striking, both $c$ and $m$ have considerably lower concentrations, yet the ECM value looks to in line. In our case the diffusion of the tumour cells into the tissue also seems to happen a littel bit too fast. The last time checkpoints strengthens our findings, showing the same behaviour with ECM being approximately the same, tumour cell density and MDE concentration being clearly lower in our experiment and invasion of tissue happening too fast, leaving the lump at the origin $x=0$ too small. \newline 
This first of all confirms the initial supposition that with changing the dimension for the simulations the results also vary. We will now adjust the parameters iteratively to make the results using two dimensions mimick the results from Anderson et. al as closely as possible. For this we will start with varying the MDE production coefficient $\alpha$, to get higher concentration values, and also change the diffusion and haptotaxis terms of the tumour cells $d_c$ and  $\gamma$, to adjust the motitlity of the tumour cells and therefore also influence the invasion speed of them into the surrouding tissue.
\begin{figure}[h!]
    \centering
    \includegraphics[width=0.8\textwidth]{resources/images/alpha_comparison.png}
    \includegraphics[width=0.8\textwidth]{resources/images/dc_comparison.png}
    \includegraphics[width=0.8\textwidth]{resources/images/gamma_comparison.png}
    \caption{Caption}
    \label{fig:replication_comparison}
\end{figure}


Figure~\ref{fig:replication_comparison} shows a comparison of the parameters $\alpha$, $d_c$ and $\gamma$ have on a specific curve. Comparing different values for $\alpha$ and their effect on the curve of the MDE concentration, shows that, especially looking at the later points in time $t=4$ and $t=8$, with values for $\alpha$ between $0.3$ and $0.4$ we will get a good approximation. The values of the original paper for the MDEs are for $t=4$ $m(0)=0.6$ and at $t=8$ $m(0)=0.7$. Fine tuning this parameter led us to $\alpha=0.35645$.\newline 
Looking at $d_c$ we chose a value of $d_c=5e-4$. Using higher values for this parameter will result in numerical instability and results that are not useable. For $\gamma$ we made a slight adjustment upwards to $\gamma=0.0055$ to have a little bit more pull on the tumour cells outward, to match the invasion speed observed in the original paper. This yields results where the small hill at the leading edge of the tumour cell concentration in the latter two points in time is a higher values for $x$, yet not as steep as for example $\gamma = 0.007$. \newline

\begin{figure}[h]
    \centering
    \includegraphics[width=\textwidth]{resources/images/2D_5e-4_1e-3_1e-3_10_0.35645_0_0.0055_1e-2_10_plot.png}
    \label{fig:2D_5e-4_1e-3_1e-3_10_0.35645_0_0.0055}
\end{figure}

These adjustments leave with the final configuration for replicating the system with the curves in figure~\ref{fig:2D_5e-4_1e-3_1e-3_10_0.35645_0_0.0055} and the parameter settings also seen in the same figure.
Slightly adjusting the haptotatic flux to $\gamma=0.0055$ yields the following results, seen in figure~\ref{fig:2D_5e-4_1e-3_1e-3_10_0.35645_0_0.0055}. comparing our final version with the original one we can see that in the second point in time, at $t=4$ in our case, the values of the three curves at the $x=0$ are nearly the same. In the original experiment the bump in the curve for the tumour concentration looks more pregnant, but this is only due to the fact, that this experiment was most likely done on the unit line, not the unit cube, it also seems that they have done a rescaling of the x-axis. The two later points in time confirm the similarity with having also nealy the same values for the three curves at $x=0$ but also their respective propagations in time look to be in line with the original experiment. 



\subsubsection{Parameter Analysis}
From the replicated results shown in figures~\ref{fig:2D_5e-4_1e-3_1e-3_10_0.35645_0_0.0055}, we saw that if we variate certain parameters the results also vary strongly. Therefore we are now going to have a look at how changing one parameter affects the output of the whole system. For this we assume the parameter values of the replicated results to be our set of baseline parameters, from there in each experiment only one parameter is changed. 
\subsubsection*{$d_c$ Variation}
The parameter analysed in this section describes the diffusion of the tumour cells and is integrated into the equations as being dependent on the laplacian of the tumour cells $\Delta c = (\frac{\partial^2 c}{\partial x^2} + \frac{\partial^2 c}{\partial y^2} + \frac{\partial^2 c}{\partial z^2})$. Leaving out the proliferation term our equation for $\frac{\partial c}{\partial t}$ also depends on $\gamma$ a coefficient for the haptotatic flux. The mathematical intuition is that if we will decrease $d_c$ we will see the effects of $\gamma$ taking over the simulation results for the $c$ curve, meaning that the tumour cells are more likely to drift outward and let themselves be pulled by the ECM concentration $e$, due to haptotaxis, leaving ony a little concentration at the center $x=0$, creating a bigger hill on the leading edge of the tumour concentration (below where $c \nabla e$ will be highest). On the other hand if we increase $d_c$ the effects of haptotaxis will diminish, the tumour cells will be subject to bigger diffusion pulling them more evenly into the tissue, there will be less of a leading hill being pulled outwards, since the diffusion will happen too fast, making this effect irrelevant. 
\begin{figure}[h]
    \centering
    \includegraphics[width=\textwidth]{resources/images/dc_variation.png}
    \caption{Plots show results for varying $d_c$ whilst keeping the other parameters constant, in the images you can see the effects of $d_c=1e-5$ in the dashed curve, $d_c=1e-1$ in the dotted curve and $d_c=5e-4$ in the solid line.}
    \label{fig:dc_comparison}
\end{figure}
Looking at both experiments in figure~\ref{fig:dc_comparison}, we can see these assumptions confirmed. The smaller $d_c$ gets the higher the influence of $\gamma$ will be and vice versa. Considering the red curves, the tumour cell density, after $t=4$ timesteps the dashed curve, describing $d_c=1e-5$, takes on a value a little higher than for the basecase of the solid curve with $d_c=5e-4$ whilst the dotted curve, showing the results for $d_c=1e-1$, has already taken on a near constant concentration throughout space. This shows that the higher the values for $d_c$ are the faster the diffusion spreading throughout space will be, where for the solid and dashed line we can see small effects of haptotaxis in the second image, nothing of this is visible for the dotted line, where the diffusion effects completly cover up the effects of haptotaxis. For the MDE concentration we can observe that with rising values for $d_c$ they also spread faster throughotu space, this is first due to influece of the current $c$ density on the motility of the MDEs and also due to the production term, meaning with faster spread throughout space we will get more even prodution of MDEs. The effect on the ECM concentration varying $d_c$ seems to be little at stage. 
Looking at the next point time at $t=40$ we can see that the differences in the curves observed previously have intensified, with examining the tumour cell concentration, yielding now completly different results. The dotted curve has not visbily changed and while the solid curve describes the base case, we can see for the dashed curve that as above mentioned here the effects of haptotaxis, with a very pointy peak at the leading edge of the tumour cells. The other two curves differ also very visibly here, showing big variation MDE and ECM concentration. The MDEs diffuse fasater throughout space the higher $d_c$ is, with the dotted curve having flatted more than the other two, where the dashed curve has the highest concentration of MDEs at the origin still. This faster spreading of tumour cells and MDEs takes effect on the ECM, showing that the ECM for the dotted curve has clearly faster decayed than the other two, though for them we can see like for the MDEs visible differences considering degradation. 
In the last image we can see another amplification of the previous mentioned effects, for the dotted curves, $d_c=1e-1$, they seem to have taken constant concentrations in space, except the green curve for the MDEs which is till higher around the origin than in outer regions, the red curve descring the tumour cells has again not changed staying constant and the ECM concentration has also been decgraded towards zero every in space. The red dashed curve shows that the tumour cells density has two clear maxima in space one at the origin and one at the leading edge, whereas the solid curve is a lot more evenly distributed throughout space. Here you can also see that the red dashed curve takes on negative values, which indicate numerical instabilities, which will only increase decreasing $d_c$. These issues are due to the nature of the solver, where both factors influcencing $c$, $d_c$ and $\gamma$ need to be in a certain range to produces reasonable results, since negative values for the tumour cell density does not make any sense. The MDE concentration is as expected higher around the origin for the lower $d_c$ values, this is due to the tumour cells also staying rather around the origin for this experiment and therefore producing more MDEs at this location in space. It is interesting to see that both effects of $c$ and $m$ comparing the dasehd and solid curves seem to have little effect on the ECM concentration, but as we saw in the third image only a little concentraion of MDEs is needed to efficiently decrease the ECM, this shows that the ECM degradation process happens so fast that minor diffrences in the MDE concentration have little effect on it and also as we can see that the major differences in the MDE curve are located around the origin, these differences seem to decrease with increasing distance.
These comparsion verifies that $d_c$ has a rather impactful influence of the system, espeically considering that the $d_c$ values of the dashed and solid curves are only seperated by a distance of $5e5$. Increasing $d_c$ results in faster diffusion and also faster spreading throughout space, but also with faster degradation of the ECM and invasion of MDE of the area.

\begin{comment}
In the first four images we can see the results for $d_c=1e-5=0.00005$, while in the it looks mostly normal with little of the secession building at the leading edge, we can see that in the third image, this secession has not only seperated itself from the main lump of cells, but has also developed a sharp peak, which would have negative consequence regarding the differentiability of the $c$ curve. In the fourth image this behaviour is even more extreme and looking closely at this simulation in ParaView, the values for $c$ take on a negative sign, which from a biological perspective does not make any sense since there can't be less than zero cells at a position in space. This indicates a numerical instability, which decreasing $d_c$ even furhter also resulted in oscilations in the $c$ curve and even more negative values for the tumour cells. This instability is due to numercial model used, which only yields useful results if $\gamma$ and $d_c$ are in a certain range. This could be a point for further investigation, inspecting how the results change if $\gamma$ and $d_c$ are both varied at the same time and finding this range $\gamma$ and $d_c$ need to be in to make sense. However looking at the other two curves $e$ and $m$ we can see that they still make sense, especially also in the first experiment with $d_c=1e-5$, $m$ at the start following where $c$ was high, exceeding $c$ at some point due to the production factor $\alpha$ and $e$ decaying where the MDEs have higher values. Interesting is that having high values for the diffusion coefficient the concentration is faster evenly distributed in space, which indicates a higher invasion pace, which in turn makes the degradation of $e$ happen a lot faster, though the MDEs have taken on a near constant niveau throughout space. This also indicates, that we don't need to have a visibly higher concentration of $m$ to degrade the ECM. 
\end{comment}

\subsubsection*{$\gamma$ Variation}
Inspecting the effects of $\gamma$ we can assume the same as for $d_c$ if we select higher values for $\gamma$ the effects of haptotaxis, pulling the tumour cells into the tissue faster, leaving no cells at the origin, taking lower values for $\gamma$, the diffusion will be superior factor for the tumour cell motility, which will result in no secession at the leading edge of the tumour cells.
\begin{figure}[h]
    \centering
    \includegraphics[width=\textwidth]{resources/images/gamma_variation.png}
    \caption{gamma variation}
    \label{fig:gamma_variation}
\end{figure}
The experiments, described in figure\ref{fig:gamma_variation} verify this behaviour, in the first one where $\gamma=0.002$ we can see the effects of haptotaxis only slightly in the third image, with a small bump for the tumour cells the farthest out. Increasing $\gamma=0.008$ we see the haptotatic effects stronger now than in the base case, the secession that gets pulled into the tissue is getting bigger and also the invasion pace at which tumour cells reach the outer regions is faster. Selecting even higher values as $\gamma = 0.01$ we see this behaviour increasing the secession at the leading edge of the tumour cells is now almost as big as the remaining part at the origin, the invasion pace and therefore also the degradation of the ECM is accellerated.
When we now take a step further and increase $\gamma$ by one potence, we can observe that the invasion pace, has gotten so high, that before finishing the simulation at $t=8$ the tumour cells have not only invaded completely up to the border but have also been pulled back towards the origin upon getting reflected at the border of the unit square and also since degradation of the ECM has not kept up with the tumour cells, leaving a situation where the tumour cells have spread further than the ECM and are now being pulled backwards again. This is therefore the first experiment which has in eight times produced results that are not longer resemble radial symmetry as seen in image ..., where we can see that the diffusion and haptotaxis properties of the tumour cells push them into the corners of the unit square, where there is still the highest concentration of ECM and afterwards back in from corners again due to haptotaxis.
Taking a look at the other curves we also get interesting results, as expected when we decrease $\gamma$ and let $d_c$ stay constant the invasion pace will decrease, having also degraded the ECM in the outer regions than the base case and also having higher MDE concentrations towards the origin since it stayed longer in these regions during the experiment. Increasing $\gamma$ over the value of the base case resulted also in hihger invasion pace, ECM degradation and lower MDE values at the origin, but this behaviour does not increase linearly. Looking at the experiment with $\gamma=0.1$ we see that though the tumour cells have reached the border regions faster, the ECM degradation could not keep up with the pace, resulting in areas where the ECM is still high, the tumour cells have already surpassed, but have not produced enough MDEs to degrade it. Where in all of the other experiments the tumour cells invaded at such a pace, that the produced MDEs in their wake were sufficient to degrade the ECMs to not pull the tumour cells back to the remaining ECMs later. This is also only experiment where the MDe curve $m$ is not monotone decreasing, with higher values at the origin and bordering regions, but lower one in between. If we were to increase $\gamma$ further we would see this behaviour mirrored, with the tumour cells spreading so fast throughout the space that we would get oscilations.
\begin{figure}[h]
    \centering
    \includegraphics[width=\textwidth]{resources/images/2D_plot.png}
    \caption{2D plot of variation}
    \label{fig:gamma_2D_plot}
\end{figure}
\begin{figure}[h]
    \centering
    \includegraphics[width=\textwidth]{resources/images/pol_comparison.png}
    \caption{Plot OVer Line Comparison Gamma}
    \label{fig:gamma_pol_comparison}
\end{figure}
\begin{figure}[h]
    \centering
    \includegraphics[width=\textwidth]{resources/images/gamma_alt_pol.png}
    \caption{gamma alternative plot over line}
    \label{fig:gamma_alt_pol}
\end{figure}

\subsubsection*{$\eta$ Variation}
\begin{figure}[h]
    \centering
    \includegraphics[width=\textwidth]{resources/images/eta_variation.png}
    \caption{eta variation}
    \label{fig:eta_variation}
\end{figure}
The parameter $\eta$ only directly influences the degrading of the ECM, which happens faster for higher $\eta$ coefficients in regions where both MDE and ECM concentration are high. Varying this parameter yet may have a high impact of the solution because, the gradient of the ECM is a deciding factor for the effects of haptotaxis on the tumour cells. \newline
The first experiment in figure~\ref{fig:eta_variation} shows that if $\eta = 0$, which means that the ECM is not degraded by the MDEs, we get completely differnt results comparing it to the base case. Not only does $\eta$ influcence the curve for the ECM $e$ but has also a high impact on both tumour cell density and MDE concentration. Due that no degrading happens $e$ stays constant all the time, with $\frac{\partial e}{\partial t} = 0$, this also means that $\nabla e$ stays also constant, we see this effect in the images, showing that $c$ does not invade further than the point where $\nabla e$ is highest, this implies that if $c$ converges towards this point it will also have its maximum at this point, which also fixes the point $c\nabla e$, what does this mean for the MDEs? After initially increasing slightly around the point $x=0$ and increasing where $c$ was high previously, it also converges with its highest concentration at $c\nabla e$. Since the motility of $c$ is highly restricted due to no degradation of the ECM, the MDEs will progress to increase $c$ takes on values higher than 0. If this simulation for $\eta=0$ will be continued we will see that the MDEs $m$ will also a curve that is not monotone decreasing.
Increasing $\eta$ to $\eta=20$ the behaviour change is not so drastically comparing it to the base case. The degradation of the ECM happens twice as fast, which results in a faster invasion pace of the tumour cells, though with decreased influcence of haptotaxis, making the hill at the leading edge smaller. After dimensionless time $t=8$ the ECMs have degrading has visibly increased, but the MDE concentration is still higher. This only makes sense since, needing a lower concentration to degrade the ECM, this process happens faster, and also since haptotactic influences are lower the concentration of tumour cells at the origin is higher, which will also produce more MDEs at the origin. 
$\eta$ has a strong influcene on all curves, if its value is lower the degradation of $e$ happens slower, slowing also $c$ and $m$'s invasion pace down. 

\subsubsection*{$d_m$ Variation}
\begin{figure}[h]
    \centering
    \includegraphics[width=\textwidth]{resources/images/dm_variation.png}
    \caption{dm variation}
    \label{fig:dm_variation}
\end{figure}
$d_m$ is the parameter describing the diffusion of the matrix degrading enzymes MDEs, it is influcenced by the second derivative of c. Looking at the equations we can expect with higher values for $d_m$ a faster degradation of $e$, since the MDEs can invade faster into the space and there are not too much MDEs needed to degrade $e$. This will then cause a faster invasion pace of $c$, because the haptotatic flux pulls heavier outward.
Setting $d_m$ to zero, so no movement of the MDEs, we see that the curve still changes, which is due to the tumour cells producing them where they are. We see that the value for $m$ around the origin is higher than 1 which is possible since their concentration could be soo high that there are more than one MDE per mesh point, although this would requiring to make the mesh finer I think. 
If we look in the other direction, setting $d_m$ to $1e-1=0.1$ we see that after already $t=0.4$ the MDEs have spread completely throughout space, from this point on, they are mostly subject to the production term yielded by the tumour cells, but as fast as they are produced, so fast they are also distributed in space, causing a semmingly equillibrium throughout space for the MDE concentration regardless of the local tumour cell density. As we saw earlier a low concentration for MDEs is needed to efficiently degrade the ECM, therefore degradation happens a lot faster here, even so fast, that the gradient of $e$ diminishes as fast that the haptotatic influcences of the tumour cells are reduced, causing tumour spread to slow down. This is contrary to our initial assumption that with higher values for $d_m$ the tumour invasion pace will also increase. This parameter therefore seems to have influence on the haptotatic effects and the overall extra cellular matrix degradation.
	

\subsubsection*{$\alpha$ Variation}
\begin{figure}[h]
    \centering
    \includegraphics[width=\textwidth]{resources/images/alpha_variation.png}
    \caption{alpha variation}
    \label{fig:alpha_variation}
\end{figure}
When we look at $\alpha$ in a range from $0.0$ to $1.0$ we can expect with growing $\alpha$ a faster degrading of the ECM and higher values form the MDEs themselves. Faster ECM degrading could mean fast invasion of the tissue of the tumour cells. As we saw in the previous comparison, the MDEs can take on values higher than one, we can also expect this here when $\alpha$ is sufficiently high. 
Looking at the experiment with $\alpha=0.0$ we can see that at the end the ECM has still much higher values than compared with the baseline experiment at dimensionless time $t=8$ and as expected the invasion pace of the tumour cells is considerably slower. 
Comparing this to the plots for $\alpha=1.0$ we can see that after already $t=4$ the MDEs have taken on a contration of greater than one at $x=0$. Overall is the degradation happening faster and therefore also the tumour invasion happens faster. In the end we are left with a clearly lower ECM concentration than for example the baseline experiment. 

\subsubsection*{$\beta$ Variation}

Looking at $\beta$ which is the parameter describing decay of the MDEs, we can assume that with varying $\beta$ the MDE curve will be lower, influcencing the ECM degrading process and therefore also the invasion pace. Since all previous experiments assumed a value of $\beta=0$ we can expect that with growing $\beta$ these effects will increase contininuously. We first of all needed to determine a range in which to experiment. Starting with a range of values between $0.1$ and $1.0$, since this is the range $\alpha$ yields reasonable results we saw that those values were much too high. Even for $\beta=0.1$ the MDEs are almost completely decayed after only $t=0.4$, looking closer at ParaView this happened after already $t=0.1$. This also affects the haptotatic effects of the tumour cell invasion, having no secession formed at the leading edge of the tumour cells. Reducing the range for $\beta$ one potence we can see the same behaviour, with a fast dacay of the MDEs therefore lower ECM degrading and slower invasion pace. Yet looking at $\beta=0.01$ we see the effects of haptotaxis now again and the ECM degrading happens here visibly faster, though the MDE concentration is generally lower. This lead us to decrease $\beta$ even further. Though if we look at $\beta=0.001$ we see that the effects of $\beta$ are barely recognizable anymore. Taking the middle of those and setting $\beta$ to $0.005$, we see all the presumed effects of slowed degradation of the ECM therefore also slowed invasion pace, yet still having the effects of haptotaxis also clearly visible.

\subsubsection*{Cross Variation}
Having done all those experiments it will be interesting to compare countering effects and supporting effects two at a time


\subsection{Two dimensional Results with Proliferation}


\subsection{Three Dimensional Results}
\subsubsection{Replicating Results}
\subsubsection{Parameter Analysis}
\subsection{Three Dimensional Simulations with Heterogenous ECM Structure}
