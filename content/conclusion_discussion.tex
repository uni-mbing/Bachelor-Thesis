\section{Conclusion and Discussion}
In this work, we conducted a parameter analysis for a numerical model in tumor development research. The aim was to investigate the impact of varying model parameters, dimensions and the structure of the extracellular matrix on simulation outcomes and provide insights into the model's behavior under different conditions. This was done to facilitate implementing this model in a real-world biological or medical scenarios, facilitating an entry point for researchers to choose the parameters for their experiments.

The results of our analysis highlight the model's sensitivity to changes in key parameters, such as the diffusion or haptotaxis coefficients on the tumor cells, the proliferation and growth rate of the tumor cells, the degradation rate of the extracellular matrix and the production and decay rates for the matrix-degrading enzymes. We observed that small variations in these parameters can lead to significant differences in simulation outcomes, indicating the importance of carefully selecting and calibrating model parameters for accurate representation of biological phenomena.

Furthermore, our study revealed the complex interplay between different parameters during cross-variating them and their effects on tumor invasion and extracellular matrix degradation. For example, it was found that increasing the haptotatic flux coefficient of the tumor cells is the critical component to controlling how many cells of a tumor invade the surrounding tissue and how many stay at the center of the simulation causing drastic changes in the rate at which the extracellular matrix is degraded. Most of the parameters studied increased the invasion pace of the tumor cells and the degradation of the extracellular matrix, when they were increased.

Moreover, the influence of spatial dimensions on simulation outcomes was shown, with simulations in higher dimensions producing qualitatively different results compared to the initial one-dimensional model. This underscores the importance of considering the spatial complexity of biological systems when designing computational models for tumor development.

Additionally, our analysis sheds light on the limitations of the current model and areas for further improvement. The model could be extended in both discrete and continuous ways, for example, by implementing biochemical effects like chemotaxis, which is done in Kolev et al.'s work~\cite{Kolev2010} or implement immune cell interactions or heterogeneity in tumor cell populations, to capture the complexity of the tumor microenvironment better.

In conclusion, our parameter analysis provides valuable insights into the behavior of the investigated numerical model in tumor development research and underscores the importance of parameter selection and model validation in Mathematical Oncology. By refining our understanding of the underlying mechanisms driving tumor progression, such models have the potential to inform therapeutic strategies and improve patient outcomes in the fight against cancer.