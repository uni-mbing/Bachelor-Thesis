\section{Theoretical Basics}
\subsection{Basics of Tumor Biology}
The body of a living creature is made up of more than 200 different 
types of cells, the coordination between the cells and their 
surroundings keep the body running. Each of these cells is built from 
the genetic information encoded in the DNA, located in the 
cells' nuclei. Though the nucleotide sequence of DNA is well 
checked and maintained throughout the cell's life, mutations still
occur that cause the changes in the DNA of a cell. These mutations may 
be of a positive, negative or neutral nature. In the case of a negative 
mutation this alternation of the DNA may cause diseases, with cancer being one 
of them. The failure of the complex system managing cell birth, proliferation, 
and cell death (apoptosis) causes cancer, resulting in an uncontrolled cell proliferation in a at 
first local area. An conglomeration of cancer cells is called a tumor. \newline
A cancer disease typically follows five stages. First the tumor initiation phase where it comes to the above 
explained genetic mutations of normal cells. The next stage is the tumor promotion stage, in which 
the mutated cells of phase one may experience further genetic alterations, with the result of 
uncontrolled growth and proliferation of the cancerous cells. The third stage is the tumor 
progression stage, where the cancerous cells progress in growing and proliferating, reaching a critical mass, they form a tumor at a 
local site of the body. Fourth comes the invasion stage, here the tumor is able to invade surrouding tissue 
and enter the blood circulation system or the lymphatic system. Next the tumor cells which have invaded 
the blood circulation of lymphatic system spread throughout the body and form new tumors. This stage is called Metastization.
To further grow the tumors need to have access to nutrient and oxygen supply. During angiogensis a tumor develops 
blod vessels of its own securing its nutritional provision. At this stage first symptoms of host may appear, enabling medical 
treatment.\newline
In our model the focus lies on the first two stages; tumor invasion and tumor progression, so we are going to have a deeper look 
at those two phases.
The tumor invasion stage is characterized by the malignant cells gaining the ability to penetrate and invade the surrouding tissuse. 
The tumor cells break through the normal tissue barrier and infiltrate neigbhoring structures. In order to do so the cancer cells produce so called 
matrix-degrading enzymes which break down the extracellular-matrix. This not only helps local spreading but also destroys otherwise healthy tissue and cells in the 
affected area. 
In the next phase the tumor progresssion stage, the tumor has grown larger and the cancerous cells take on more aggressive behaviour, by invading the surrounding area further.
Whilst they keep growing uncontrolled they are also affected by further genetic instabilities, which lead to more mutations among the tumor cells, resulting possibly in the 
development of resistent cancer cells. Already in this stage the affected area is exposed to heavy tissue damage and functional disabilities.\newline
The most important factors influencing those two phases are the genetic dispoistions of the tumor cells towards proliferation and the evasion 
of apoptosis, which increase the invasive potential. Another important factor is geometry of the extracellular matrix, as well as the exact macromolcules 
which make it up. A strong immune biological defense reaction also helps the body defend against the spreading of the cancer cells, so evasion of detection and 
destruction of the tumor cells plays a key role for the first stages. To invade the affected area the malignant cells need to be able to move freely and fastly. In order 
to do so cancer cells can gain the ability to lose adhestion properties which healthy cells have, to allow migrating into surrouding tissue. 


\subsection{Mathematical Methods in Oncology}
Mathematical Methods and Models in Oncology play a cruical role in analysing, understanding and predicting cancer development. 
Since the obejctive of this research underlies complex and intricate biological systems and mechanisms, there exist many models, which find their repsective application in 
many distinct areas of this research field. These methods can be coarsely divided into three sections; continuous, discrete and hybrid models. 
For describing tumor growth, exponential and logisitc growth models are often used, the later allowing limiting factors to play a role during modelling.  
These methods are a subclass of the differential equations approach which base their functionality on a ordinary or partial differential equation, studying the continuous approach. 
Like in our model they are not limited to consist of one equation but can of many, therefore also incorporating limiting or accellerating factors. These models in generals deal with 
continuous parameters like densities, or fluid concentrations, for example spacial and temporal nutritional supply or drug concentration, as well as their effects on the affected area over time. 
Discrete models use a agent-based approach, where the participating individual entities are modeled as objects which can interact on their environment, this means for example cell-cell interaction or 
cell-tissue interaction. This enables researchers to focus on biological effects during modelling. With these approaches we can also simulate genetic and evolutionary events. For example studying the genetic alternations of tumor cells.\newline 
Hybid models combine both aforementioned methods, of using continuous and discrete models. Like in the model proposed by Franssen et al. \cite*{franssen_mathematical_2019}, these apporoaches allow to incorporate the exactness of continuous models with the wide range of biological effects of discrete models. \newline
But no all models try to model tumor growth, there others concerning for example optimality regarding drug dossages or radition exposition, offering personalized treatment, or Machine Learining and Data Mining methods analysing large datasets, to identify patters and predict outcomes. The later method may be used in all kinds of applications, for example spacial or temporal cancer development but also for drug dosage optimization for individual patients. 
Putting all these methods together gives us an powerful toolbox to simulate and understand cancer biology. Like the last years have shown they are applied in a wide range, offering insight for all areas of cancer research.
Therefore it is important not only to come up with methods but to also evaluate their usefulness and meaningfulness regarding different areas of research.